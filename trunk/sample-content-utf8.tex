

%\categorynumber{} % 分类采用《中国图书资料分类法》
%\UDC{} %《国际十进分类法UDC》的类号
\secretlevel{公开} %学位论文密级分为“公开”、“内部”、“秘密”和“机密”四种
\studentid{050962} %学号要完整,前面的零不能省略。

\title{东南大学~RoboCup~机器人训练基地}
\englishtitle{RoboCup Research Group of Southeast University}
\subtitle{——到2050年战胜人类足球世界冠军}
\englishsubtitle{---to beat the human soccer champion by the year 2050}
\author{东南大学}
\englishauthor{SEU}

\advisor{许映秋}{教授}{Xu Yingqiu}{Prof.}
\coadvisor{谈英姿}{副教授}{Tan Yingzi}{Associate Prof.}
\major{机器人}
\submitdate{2007年12月20日}{December 20, 2007}
\defenddate{2050年7月7日}
\authorizedate{2050年7月7日}
\date{\today}


\maketitle

\chapter{RoboCup简介}

机器人足球赛,顾名思义,就是制造和训练机器人进行足球比赛。通过这种方式提高人工智能领域、机器人领域的研究水平。

RoboCup是一个国际性的研究和教育组织,它通过提供一个标准问题(机器人足球比赛)促进人工智能和智能机器人的研究。由于这个平台可以集成并检验很大范围内的技术,同时也可以作为面向教育的集成性项目,故而引起越来越多国家的重视。

RoboCup机器人足球赛涉及人工智能、机器人学、通讯、传感、精密机械和仿生材料等诸多领域的前沿研究和技术集成,实际上是高技术的对抗赛。此项赛事有严格的比赛规则,融趣味性、观赏性、科普性为一体,比赛结果可以从一个侧面反映一个国家信息与自动化领域基础研究和高技术发展的水平。


\chapter{东南大学RoboCup机器人训练基地历程}

东南大学RoboCup仿真组在学校的大力支持下,由机械工程学院许映秋副院长(教授)、自动化学院谈英姿副教授牵头组织,于2003年5月份开始组队,取名东南虎踞队。在经过短短3个月的准备后,东南虎踞队成功参加了在北京举行的2003“马斯特杯”中国RoboCup机器人大赛。从北京参赛归来后,东南大学RoboCup仿真组人员进行了调整,组成了以博士生和硕士生为主的研究团队,全面展开RoboCup仿真的相关研究工作。

2004年11月,为进一步发展东南大学机器人研究水平和提高东南大学国际知名度,在校领导的多方关心下,东南大学RoboCup仿真训练基地在河海院正式挂牌成立!基地的成立为我校深入研究机器人特别RoboCup相关领域提供了广阔的平台,营造了良好的学术氛围。同年,基地举办了首届东南大学RoboCup联赛,全校同学积极参加,并取得圆满成功!此次比赛开拓了同学们的视野,提高了同学们的科技创新能力,引起了全校师生的强烈反响,同时也扩大了RoboCup在东南大学的影响。

2005年2月东南大学虎踞队参加RoboCup国际资格赛,经过老师和同学们的努力,获得参加在日本举行的RoboCup-2005国际机器人足球锦标赛的资格。2005年7月,东南大学第一次参加了RoboCup国际机器人足球锦标赛,通过此次比赛和交流,积累了宝贵的经验,掌握了RoboCup研究的最新动态。在系统分析国内外RoboCup研究现状和发展趋势后,基地决定启动并立即着手进行RoboCup3D仿真的研究项目,此举使得我校在RoboCup机器人足球训练方面的研究紧跟国际步伐。

2006年6月,经过近一年认真研究和精心准备后,东南大学虎踞龙蟠队在德国不来梅举行的RoboCup-2006国际机器人足球锦标赛上,取得了3D仿真组第六名和2D仿真组第十名的好成绩,表明东南大学RoboCup的研究水平有了长足的进步,也标志着我校相关研究进入一个新的阶段。

2006年10月,为了提高我国社会应急救助系统的相关研究水平,在基地老师的倡导下,东南大学营救仿真队伍诞生,由此,扩大了基地的规模,拓展了基地的研究领域,为研究成果的转化提供了平台。

2007年7月,RoboCup-2007国际机器人足球锦标赛在美国亚特兰大举行,我校3D仿真组和营救仿真组分别取得第三名和第五名的好成绩。本次比赛有力地展示了东南大学RoboCup机器人训练基地的研究成果,体现了我校本科生SRTP项目、研究生自主创新计划、课程教学与前沿研究相结合等优良传统和教改举措的优势。

\section{取得的成绩}
东南大学RoboCup机器人训练基地在参加的历次大赛中取得了令人瞩目的成绩,这是与老师和同学们的辛勤劳动密不可分的。

历届参赛回顾:

\subsection{RoboCup国际机器人足球锦标赛}
\begin{description}
\item[2007亚特兰大(美国)] 3D仿真组季军,营救仿真组第五名(首次参赛)。
\item[2006 布莱梅(德国)] 3D仿真组第六名,2D仿真组第十名。
\item[2005 大阪(日本)] 2D仿真组十六强。
\end{description}

\subsection{全国机器人大赛}
\begin{description}
\item[2007 济南] 3D仿真组冠军,Rescue仿真组亚军。
\item[2006 苏州(首届中国公开赛)] 3D仿真组亚军,2D仿真组十六强。
\item[2005 常州] 3D仿真组第七名(首次参赛),2D仿真组第十名。
\item[2004 广州] 2D仿真组第七名。
\item[2003 北京] 2D仿真组三等奖(首次参赛)。
\end{description}

\subsection{RoboCup公开赛}

\subsubsection{2007}
\begin{description}
\item[伊朗公开赛] 3D仿真组冠军。
\end{description}

\subsubsection{2006}
\begin{description}
\item[伊朗公开赛] 3D仿真组八强。
\item[荷兰公开赛] 3D仿真组亚军,2D仿真组季军。
\end{description}

\subsubsection{2005}
\begin{description}
\item[伊朗Isfahan大学举办的AI-Games] 3D仿真组第四名。
\item[美国公开赛] 2D仿真组季军。
\end{description}

\subsection{江苏省机器人大赛}
\begin{description}
\item[第二届(2006)] 3D仿真组冠军。
\item[第一届(2004)] 2D仿真组亚军。
\end{description}

\section{组织活动}
其间为了培养同学们的团队意识,基地曾经组织数次户外活动,如到十月军校进行能力拓展,到栖霞山秋游,到天文台考察,到太阳宫戏水,到中国科技大学访问交流,组织暑期集训营等。这些活动活跃了团队气氛,增加了团队凝聚力。


\section{基地现状}
数年来基地已经是学校培养创新人才的重要机构,一大批优秀人才从这里走出去,获得学校和社会一致好评,他们中包括博士1人,硕士11人,本科生数十人。目前东南大学RoboCup机器人训练基地正向着更高的目标前进,主要研究方向有类人机器人足球仿真和营救仿真。

基地现由两名教授指导,目前的主要成员包括硕士生5人,本科生7人。

\backmatter     % 开始正文之后的部分

%\bibliographystyle{unsrt}%plain
%\bibliography{content/reference}

%%% Local Variables: 
%%% mode: TeX-PDF
%%% TeX-master: "sample-utf8"
%%% End: 
